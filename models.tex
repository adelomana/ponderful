% Options for packages loaded elsewhere
\PassOptionsToPackage{unicode}{hyperref}
\PassOptionsToPackage{hyphens}{url}
%
\documentclass[
]{article}
\usepackage{amsmath,amssymb}
\usepackage{iftex}
\ifPDFTeX
  \usepackage[T1]{fontenc}
  \usepackage[utf8]{inputenc}
  \usepackage{textcomp} % provide euro and other symbols
\else % if luatex or xetex
  \usepackage{unicode-math} % this also loads fontspec
  \defaultfontfeatures{Scale=MatchLowercase}
  \defaultfontfeatures[\rmfamily]{Ligatures=TeX,Scale=1}
\fi
\usepackage{lmodern}
\ifPDFTeX\else
  % xetex/luatex font selection
\fi
% Use upquote if available, for straight quotes in verbatim environments
\IfFileExists{upquote.sty}{\usepackage{upquote}}{}
\IfFileExists{microtype.sty}{% use microtype if available
  \usepackage[]{microtype}
  \UseMicrotypeSet[protrusion]{basicmath} % disable protrusion for tt fonts
}{}
\makeatletter
\@ifundefined{KOMAClassName}{% if non-KOMA class
  \IfFileExists{parskip.sty}{%
    \usepackage{parskip}
  }{% else
    \setlength{\parindent}{0pt}
    \setlength{\parskip}{6pt plus 2pt minus 1pt}}
}{% if KOMA class
  \KOMAoptions{parskip=half}}
\makeatother
\usepackage{xcolor}
\usepackage[margin=1in]{geometry}
\usepackage{color}
\usepackage{fancyvrb}
\newcommand{\VerbBar}{|}
\newcommand{\VERB}{\Verb[commandchars=\\\{\}]}
\DefineVerbatimEnvironment{Highlighting}{Verbatim}{commandchars=\\\{\}}
% Add ',fontsize=\small' for more characters per line
\usepackage{framed}
\definecolor{shadecolor}{RGB}{248,248,248}
\newenvironment{Shaded}{\begin{snugshade}}{\end{snugshade}}
\newcommand{\AlertTok}[1]{\textcolor[rgb]{0.94,0.16,0.16}{#1}}
\newcommand{\AnnotationTok}[1]{\textcolor[rgb]{0.56,0.35,0.01}{\textbf{\textit{#1}}}}
\newcommand{\AttributeTok}[1]{\textcolor[rgb]{0.13,0.29,0.53}{#1}}
\newcommand{\BaseNTok}[1]{\textcolor[rgb]{0.00,0.00,0.81}{#1}}
\newcommand{\BuiltInTok}[1]{#1}
\newcommand{\CharTok}[1]{\textcolor[rgb]{0.31,0.60,0.02}{#1}}
\newcommand{\CommentTok}[1]{\textcolor[rgb]{0.56,0.35,0.01}{\textit{#1}}}
\newcommand{\CommentVarTok}[1]{\textcolor[rgb]{0.56,0.35,0.01}{\textbf{\textit{#1}}}}
\newcommand{\ConstantTok}[1]{\textcolor[rgb]{0.56,0.35,0.01}{#1}}
\newcommand{\ControlFlowTok}[1]{\textcolor[rgb]{0.13,0.29,0.53}{\textbf{#1}}}
\newcommand{\DataTypeTok}[1]{\textcolor[rgb]{0.13,0.29,0.53}{#1}}
\newcommand{\DecValTok}[1]{\textcolor[rgb]{0.00,0.00,0.81}{#1}}
\newcommand{\DocumentationTok}[1]{\textcolor[rgb]{0.56,0.35,0.01}{\textbf{\textit{#1}}}}
\newcommand{\ErrorTok}[1]{\textcolor[rgb]{0.64,0.00,0.00}{\textbf{#1}}}
\newcommand{\ExtensionTok}[1]{#1}
\newcommand{\FloatTok}[1]{\textcolor[rgb]{0.00,0.00,0.81}{#1}}
\newcommand{\FunctionTok}[1]{\textcolor[rgb]{0.13,0.29,0.53}{\textbf{#1}}}
\newcommand{\ImportTok}[1]{#1}
\newcommand{\InformationTok}[1]{\textcolor[rgb]{0.56,0.35,0.01}{\textbf{\textit{#1}}}}
\newcommand{\KeywordTok}[1]{\textcolor[rgb]{0.13,0.29,0.53}{\textbf{#1}}}
\newcommand{\NormalTok}[1]{#1}
\newcommand{\OperatorTok}[1]{\textcolor[rgb]{0.81,0.36,0.00}{\textbf{#1}}}
\newcommand{\OtherTok}[1]{\textcolor[rgb]{0.56,0.35,0.01}{#1}}
\newcommand{\PreprocessorTok}[1]{\textcolor[rgb]{0.56,0.35,0.01}{\textit{#1}}}
\newcommand{\RegionMarkerTok}[1]{#1}
\newcommand{\SpecialCharTok}[1]{\textcolor[rgb]{0.81,0.36,0.00}{\textbf{#1}}}
\newcommand{\SpecialStringTok}[1]{\textcolor[rgb]{0.31,0.60,0.02}{#1}}
\newcommand{\StringTok}[1]{\textcolor[rgb]{0.31,0.60,0.02}{#1}}
\newcommand{\VariableTok}[1]{\textcolor[rgb]{0.00,0.00,0.00}{#1}}
\newcommand{\VerbatimStringTok}[1]{\textcolor[rgb]{0.31,0.60,0.02}{#1}}
\newcommand{\WarningTok}[1]{\textcolor[rgb]{0.56,0.35,0.01}{\textbf{\textit{#1}}}}
\usepackage{graphicx}
\makeatletter
\def\maxwidth{\ifdim\Gin@nat@width>\linewidth\linewidth\else\Gin@nat@width\fi}
\def\maxheight{\ifdim\Gin@nat@height>\textheight\textheight\else\Gin@nat@height\fi}
\makeatother
% Scale images if necessary, so that they will not overflow the page
% margins by default, and it is still possible to overwrite the defaults
% using explicit options in \includegraphics[width, height, ...]{}
\setkeys{Gin}{width=\maxwidth,height=\maxheight,keepaspectratio}
% Set default figure placement to htbp
\makeatletter
\def\fps@figure{htbp}
\makeatother
\setlength{\emergencystretch}{3em} % prevent overfull lines
\providecommand{\tightlist}{%
  \setlength{\itemsep}{0pt}\setlength{\parskip}{0pt}}
\setcounter{secnumdepth}{-\maxdimen} % remove section numbering
\ifLuaTeX
  \usepackage{selnolig}  % disable illegal ligatures
\fi
\usepackage{bookmark}
\IfFileExists{xurl.sty}{\usepackage{xurl}}{} % add URL line breaks if available
\urlstyle{same}
\hypersetup{
  pdftitle={R Notebook},
  hidelinks,
  pdfcreator={LaTeX via pandoc}}

\title{R Notebook}
\author{}
\date{\vspace{-2.5em}}

\begin{document}
\maketitle

This is an \href{http://rmarkdown.rstudio.com}{R Markdown} Notebook.
When you execute code within the notebook, the results appear beneath
the code.

Try executing this chunk by clicking the \emph{Run} button within the
chunk or by placing your cursor inside it and pressing
\emph{Cmd+Shift+Enter}.

Load libraries

\begin{Shaded}
\begin{Highlighting}[]
\FunctionTok{library}\NormalTok{(nlme)}
\end{Highlighting}
\end{Shaded}

read data

\begin{Shaded}
\begin{Highlighting}[]
\NormalTok{df }\OtherTok{=} \FunctionTok{read.csv}\NormalTok{(}\StringTok{\textquotesingle{}/Users/adrian/onedrive\_uvic/OneDrive {-} Universitat de Vic/Data PhyChe/PhyChe\_XY.csv\textquotesingle{}}\NormalTok{)}
\end{Highlighting}
\end{Shaded}

\begin{Shaded}
\begin{Highlighting}[]
\NormalTok{df\_no\_nas }\OtherTok{\textless{}{-}} \FunctionTok{na.omit}\NormalTok{(df)}
\end{Highlighting}
\end{Shaded}

this is a description

\begin{Shaded}
\begin{Highlighting}[]
\NormalTok{m1 }\OtherTok{=} \FunctionTok{lme}\NormalTok{(}
  \AttributeTok{fixed =}\NormalTok{ TP }\SpecialCharTok{\textasciitilde{}}\NormalTok{ Imp\_grassland\_100 }\SpecialCharTok{+}\NormalTok{ Open\_nature\_100 }\SpecialCharTok{+}\NormalTok{ Forest\_100 }\SpecialCharTok{+}\NormalTok{ Pasture\_100 }\SpecialCharTok{+}\NormalTok{ Urban\_100 }\SpecialCharTok{+}\NormalTok{ Other\_100 }\SpecialCharTok{+}\NormalTok{ Forest\_5  }\SpecialCharTok{+}\NormalTok{ Animals\_cont }\SpecialCharTok{+}\NormalTok{ Area }\SpecialCharTok{+}\NormalTok{ Depth }\SpecialCharTok{+}\NormalTok{ T1 }\SpecialCharTok{+}\NormalTok{ P1, }
  \AttributeTok{random =} \FunctionTok{list}\NormalTok{(}\SpecialCharTok{\textasciitilde{}}\DecValTok{1}\SpecialCharTok{|}\NormalTok{PondCode, }\SpecialCharTok{\textasciitilde{}}\DecValTok{1}\SpecialCharTok{|}\NormalTok{Country,}\SpecialCharTok{\textasciitilde{}} \DecValTok{0}\SpecialCharTok{+}\NormalTok{Imp\_grassland\_100}\SpecialCharTok{|}\NormalTok{Country,}\SpecialCharTok{\textasciitilde{}} \DecValTok{0}\SpecialCharTok{+}\NormalTok{Open\_nature\_100}\SpecialCharTok{|}\NormalTok{Country,}\SpecialCharTok{\textasciitilde{}} \DecValTok{0}\SpecialCharTok{+}\NormalTok{Forest\_100}\SpecialCharTok{|}\NormalTok{Country,}\SpecialCharTok{\textasciitilde{}} \DecValTok{0}\SpecialCharTok{+}\NormalTok{Pasture\_100}\SpecialCharTok{|}\NormalTok{Country,}\SpecialCharTok{\textasciitilde{}} \DecValTok{0}\SpecialCharTok{+}\NormalTok{Urban\_100}\SpecialCharTok{|}\NormalTok{Country,}\SpecialCharTok{\textasciitilde{}} \DecValTok{0}\SpecialCharTok{+}\NormalTok{Other\_100}\SpecialCharTok{|}\NormalTok{Country),}
  \AttributeTok{method =} \StringTok{"ML"}\NormalTok{,}
  \AttributeTok{data =}\NormalTok{ df, }
  \AttributeTok{na.action =}\NormalTok{ na.omit)}
\end{Highlighting}
\end{Shaded}

\begin{Shaded}
\begin{Highlighting}[]
\FunctionTok{plot}\NormalTok{(m1)}
\end{Highlighting}
\end{Shaded}

\includegraphics{models_files/figure-latex/unnamed-chunk-5-1.pdf}

this is some text

\end{document}
